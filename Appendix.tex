\documentclass[main.tex]{subfiles}
\begin{document}

\chapter{Appendix}
\section{Visualization of Quantum States}
In this section, the Bloch sphere and Wigner function are explained as visualization techniques.

\subsection{Bloch sphere}
\begin{figure}[H]
    \centering
    \includegraphics[width=0.3\textwidth]{figs/bloch_sphere.png}
    \caption{Representation of an abritrary pure quantum state on the Bloch sphere.
    \\ Source:~\cite{glosser.ca_english:_2012} (CC BY-SA 3.0).
    }%
    \label{fig:bloch_sphere}
\end{figure}

The pure state of a qubit can be visualized on the surface of a unit sphere with the following parametrization
\begin{equation}
    \ket{\psi} = \cos(\frac{\theta}{2})\ket{0} + \ex^{i\phi}\sin(\frac{\theta}{2})\ket{1}
\end{equation}
where \( \theta \) and \( \phi \) are the angles shown in \cref{fig:bloch_sphere}.
The ground state \(\ket{0}\) is located on the ``north pole'' and the excited state \(\ket{1}\) on the ``south pole''.

Further, a handy trick to visualize the qubit when it has more levels than three is to ``project'' the state to the \(\qty{\ket{0},\ket{1}}\) basis with \( \op{0} + \op{1} \) and then plot it on the Bloch sphere.
This means that the subspace which is not spanned by the first two levels will be represented by the inside of the sphere.
For example, the state \(\ket{2}\) will lie right in the center of the sphere while \(\qty(\ket{1}+\ket{2})/\sqrt{2}\) will lie halfway between the center and the south pole.

%\subsection{Density matrices and Hinton diagrams}
%Even tough this thesis will only deal with pure quantum states \(\ket{\Psi}\) it can be interesting to look at the density matrix \(\op{\Psi}\).
%This density matrix can be visualized in a Hinton diagram which exposes the coefficients of the basis states of \(\ket{\Psi}\) in a visual way.
%This is a good way for quickly judging the quality of a quantum state.
% TODO Explain the Hinton diagram and density matrix

\subsection{Wigner function}
The wigner function can be used to visualize quantum states and processes.
It is a quasiprobability distribution where states exhibiting quantum phenomena will have negative values, impossible for classical states.
% TODO Explain why the wigner function is good for visualization.

\newpage
\section{Cat Code Supplementary Figures}
\subsection{Wigner Function of Resonator Target States}%
\label{sec:resonator-target-wigner}
The six target states in the cat code basis are shown in \cref{fig:cat-resonator-wigner-targets} with \(\alpha=2\) and a Hilbert space size of \(N_r=8\).
Looking at \(\ket{0}\) we can see two lobes with an interference fringe in between them.
For large enough \(N_r\) the lobes will appear circular.
Compared to the basis states, the superposition states exhibit even more complex shapes.

\begin{figure}[ht]
	\centering
	\foreach \n/\capn [count=\ni] in {{0}/{\ket{0}},{5}/{\ket{1}},{3}/{(\ket{0}+\ket{1})/\sqrt{2}},{1}/{(\ket{0}-\ket{1})/\sqrt{2}},{2}/{(\ket{0}+i\ket{1})/\sqrt{2}},{4}/{(\ket{0}-i\ket{1})/\sqrt{2}}}{
		\subcaptionbox{\(\capn{}\)}{
			\centering
			\includegraphics[width=0.30\textwidth]{figs/cat_wigner_targets_\n.png}
		}%
		\ifnum\ni=6%
				%
		\else%
			\hfill
		\fi%
	}
	\caption{%
	Wigner function of the six target states in the cat code basis with \(\alpha = 2\) and \(N_r=8\).
	}%
	\label{fig:cat-resonator-wigner-targets}
\end{figure}

\newpage
\subsection{Occupation Dynamics of Resonator}%
\label{sec:resonator-occupation}
The occupation dynamics for the resonator are plotted in \cref{fig:cat-resonator-occupation}.
Although they are hard to read and no conclusions are drawn from them in the thesis they are included in the appendix as reference for further analysis.

\begin{figure}[ht]
	\centering
	\foreach \n/\capn [count=\ni] in {{0}/{\ket{0}},{5}/{\ket{1}},{3}/{(\ket{0}+\ket{1})/\sqrt{2}},{1}/{(\ket{0}-\ket{1})/\sqrt{2}},{2}/{(\ket{0}+i\ket{1})/\sqrt{2}},{4}/{(\ket{0}-i\ket{1})/\sqrt{2}}}{
		\subcaptionbox{Transfer of \(\capn{}\)}{
			\centering
			\setlength\figureheight{20em}
			\setlength\figurewidth{0.45\textwidth}
			\input{figs/cat_res_occ_60,0_\n.tikz}
		}%
		\ifnum\ni=6%
				%
		\else%
			\hfill
		\fi%
	}
	\caption{Resonator level occupation over time for the 6 state transfers.}%
	\label{fig:cat-resonator-occupation}
\end{figure}


\newpage
\section{Jupyter Notebooks and Optimization Data}
To ease the reproducibility of the experiments the Jupyter notebooks used in this thesis and majority of resulting data are available at \url{https://github.com/JohanWinther/cat-state-encoding}.
Although the project

\end{document}
