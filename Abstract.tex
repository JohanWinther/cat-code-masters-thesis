\documentclass[main.tex]{subfiles}
\begin{document}
Quantum computing has gained a lot of interest in recent years and commercial products are just now entering the market.
However one of the main challenges in realising a quantum computer is noise and one key technology to remedy this is quantum error correction (QEC).
One way of performing QEC exploits the storage of the quantum information of a qubit in a resonator as cat codes.
To do this one needs to apply encoding pulses to the coupled qubit-resonator system in order to perform a state transfer from the qubit to the resonator.
These pulses need to be numerically obtained by simulation.

This thesis studies the potential of using a gradient-based optimization method, the so called Krotov's method, to numerically optimize encoding pulses for encoding arbitrary qubit states in cat codes.
The Python package Krotov, a package for quantum optimal control using the method, is first used to perform state evolution from \(\ket{0}\) to \(\ket{1}\) and \(\ket{0}\) to \(\ket{2}\) of an anharmonic resonator in order to familiarise with the package and optimal control in general.
It is shown that, assuming a maximum drive amplitude and no dissipation, the method can realise a \(\ket{0}\) to \(\ket{1}\) evolution with fidelity \( F > 0.99999\) and a total pulse length of only \ns{10.75}.
For the \(\ket{0}\) to \(\ket{2}\) evolution a total pulse length of \ns{30} is needed to reach the same fidelity.
Finally, the Krotov is used to optimize pulses for transferring qubit states into the resonator as cat codes.
The transfer of six states were simultaneously optimized in order to approximate a unitary which transfers an arbitrary qubit state into the resonator as a cat code.
Using mostly experimentally realistic parameters, it is shown that the method can optimize pulses which realise the encoding of arbitrary qubit states to cat codes with a fidelity of at least \( F > 0.998900\).
Although plenty of challenges still remain to prove this can be done in experiments, the results points to the Krotov package as a viable tool for encoding pulse optimization.
\end{document}