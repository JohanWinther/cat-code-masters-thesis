\documentclass[main.tex]{subfiles}
\begin{document}
Quantum computing has gained a lot of interest in recent years and commercial products are just now entering the market.
However one of the main challenges in realising a quantum computer is noise and one key technology to remedy this is quantum error correction (QEC).
One way of performing QEC is to store the quantum information of a qubit, while it's idle, as special basis states in a resonator.
To do this one needs to apply encoding and decoding pulses to the coupled qubit-resonator system to do a state transfer.
These pulses are hard or perhaps impossible to solve for analytically, which means they need to be numerically obtained by simulation.

This thesis studies the prospects of using Krotov's method (gradient ascent) to numerically optimize encoding pulses for state transfer.
The Python package Krotov, a package for quantum optimal control using Krotov's method, is used to perform state transfers |0> to |1> and |0> to |2> of an anharmonic resonator.
It is shown that, assuming a maximum drive amplitude and no dissipation, the method can realise a |0> to |1> transfer with fidelity F > 0.99999 with a total pulse length of only 10.75 ns.
For the |0> to |2> transfer a total pulse length of 30 ns is needed to reach the same fidelity.
Thus it is concluded that state transfers using Krotov for non-coupled systems is viable.

However for coupled systems there is great difficulty in using Krotov, as the package assumes a time-independent control Hamiltonian.
A potential workaround for this problem is also presented.
\end{document}