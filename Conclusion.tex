\documentclass[main.tex]{subfiles}
\begin{document}

\chapter{Conclusions}
In this chapter the concluding remarks of the thesis will be presented.

\section{Conclusive Discussion of the Use of \krotov{}}
\label{sec:general}
\krotov{} provides an easy-to-use interface for setting up optimization problems.
This is the strongest argument for using this package as the features are unrivaled compared to other optimization packages.
Firstly, QuTiP~\cite{johansson_qutip_2013} can optimize with the GRAPE and CRAB algorithms, but there are no provided interfaces for
(1) saving optimizations to a file,
(2) continuing optimization after they are stopped,
(3) optimizing for multiple control fields and objectives,
(4) using custom convergence criteria and 
(5) modifying the pulses after every iteration (for amplitude constraints).
These are all critical features which \krotov{} has.
Secondly, GOAT~\cite{machnes_gradient_2015} is a new promising algorithm for quantum optimal control, but has no Python package available yet and the MATLAB implementation has a steep learning curve.
If or when the GOAT Python implementation becomes available it would be interesting to compare the feature set and the actual algorithm by testing comparing the performance on the same optimization objective.

The slow convergence of the optimization could pose a problem for larger systems, however there are some techniques for speeding up the optimization which are presented in \Cref{sec:future-work}.
If all else fails one could always use more computing power.

\section{Conclusion of Thesis}
In this thesis, we have demonstrated that Krotov's method efficiently realises the \(\ket{0}\rightarrow\ket{1}\) evolution for various pulse lengths when supplying guess pulses close to \(\pi\)-pulses.
It can also find non-trivial solutions for shorter pulse lengths when an amplitude constraints is present.
\krotov{} also realises a \(\ket{0}\rightarrow\ket{2}\) evolution for various pulse lengths.
This proves that \krotov{} has potential to be used for non-trivial systems.

Finally and most importantly, we have shown that \krotov{} allows to find an optimal pulse solution that allows to control a coupled system composed of a qubit and a resonator, such that the qubit state is dynamically transferred to the resonator by the solution pulse.
This procedure is the basis for encoding qubit quantum information into an error correcting code such as a cat code.
As the method could realise the transfer for arbitrary qubit states, there is good reason to conclude that \krotov{} has the potential to be a valid tool in realising QEC with cat codes.
Even though finding the optimal solution is computationally heavy, the pulses only need to be found once in order to be used for encoding.
Although many challenges are still unsolved, this proof of concept gives confidence for further research into using \krotov{} for finding encoding pulses.

\section{Future work}%
\label{sec:future-work}
As stated in the previous discussions, there are many ways to extend this work.
Suggestions for future work are to:
\begin{itemize}
    \item fix the unrealistic sample rate,
    \item include dissipation/noise,
    \item penalize undesired states,
    \item perform ensemble optimization for different phases of the coupling pulses
    \item implement a problem-specific system propagator using Cython\footnote{More info: \url{https://krotov.readthedocs.io/en/latest/09_howto.html\#how-to-maximize-numerical-efficiency}.} for speedup and
    \item find pulses for other bosonic codes.
\end{itemize}
To encourage the reproduction of these results and further research a link to the source code is given in \Cref{sec:jupyter}.

\end{document}
