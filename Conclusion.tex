\documentclass[main.tex]{subfiles}
\begin{document}

\chapter{Conclusion}
In this chapter some concluding remarks will be presented.
Krotov's method efficiently realises the \(\ket{0}\rightarrow\ket{1}\) transfer for various pulse lengths when supplying guess pulses close to \(\pi\)-pulses.
It can also find non-trivial solutions for shorter pulse lengths, when an amplitude constraints is present.
This proves that \krotov{} has potential to be used for non-trivial systems.

As the method could realise the transfer for arbitrary qubit states, there is good reason to conclude that \krotov{} has the potential to be a valid tool in realising QEC with cat codes.
Even though it can be slow the pulses only need to be found once in order to be used for encoding.
Although many challenges are still unsolved, this proof of concept gives confidence for further research into using \krotov{} for encoding pulse optimization.

\section{Future work}
\label{sec:future-work}
As stated in the previous chapter, there are many ways to extend this work and solve these problems.
Suggestions for future work are to:
\begin{itemize}
    \item fix the unrealistic sample rate,
    \item include dissipation/noise,
    \item penalize undesired states,
    \item perform ensemble optimization for different phases of the coupling pulses
    \item use the dispersive regime coupling term,
    \item implement a problem-specific system propagator using Cython for speedup and
    \item find pulses for other bosonic codes.
\end{itemize}
To encourage the reproduction of these results and further research a link to the source code is given in \Cref{sec:jupyter}.

\end{document}
