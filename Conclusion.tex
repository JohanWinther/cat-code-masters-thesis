\documentclass[main.tex]{subfiles}
\begin{document}

\chapter{Conclusion}

In conclusion, Krotov's method efficiently realises the \(\ket{0}\rightarrow\ket{1}\) transfer for various pulse lengths when supplying guess pulses close to \(\pi\)-pulses.
It can also find non-trivial solutions for shorter pulse lengths, when an amplitude constraints is present,


\section{Using \krotov{} For Cat Code Encoding Pulse Generation}

Together with improvements in the model and techniques for speeding up the optimization this could be a valid tool in realising QEC with cat codes as the pulses only need to be found once in order to be used for encoding.

\section{Future work}
\label{sec:future-work}

Improving the speed of the optimizations in this thesis could be done by
\begin{itemize}
    \item parallelizing the simultaneous objectives (\krotov{} provides an interface for this),
    \item using the dispersive regime approximation and
    \item implement a problem-specific system propagator using Cython.
\end{itemize}

Dispersive regime
Penalize higher states
Include dissipation (Liouvillian)
Ensemble optimization for different phases of the coupling pulses

Perhaps one could find an optimization method that is faster at fine-tuning and combine this with Krotov's method.

\end{document}
