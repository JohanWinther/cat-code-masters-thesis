\documentclass[main.tex]{subfiles}
\begin{document}

\chapter{Results}
The results of the numerical experiments will be presented in this chapter. For both optimisation targets the optimised pulses, their properties and the final evolved state will be presented.


\section{\texorpdfstring{\boldmath\(\ket{0}\rightarrow\ket{1}\)}{0 -> 1} state transfer}
In \cref{fig:3d-optim-ge} the fidelity during all optimisation runs are plotted.
For pulse lengths longer than \SI{15}{\nano\second} the fidelity starts at values close to the goal (\(F>0.9\)) and the number of iterations is low (less than 85 iterations).
In contrast, pulses shorter than \SI{15}{\nano\second} start at lower fidelities while the number of iterations are roughly one order of magnitude larger with no clear pattern.

To give more detail, the starting fidelity and optimised fidelity is plotted over pulse length in \cref{fig:fidelity-length-ge}.
The optimisations where the fidelity goal was not reached, pulse lengths equal to and below \SI{10.50}{\nano\second}, are marked with stars.
%The drop-off in fidelity for shorter pulses is linear.

\begin{figure}
    \centering
    \includegraphics[width=\linewidth]{figs/3d-optim-ge.png}
    \caption{Fidelity during optimisations for every pulse length (ns). The different colors help distinguish the lines.}
    \label{fig:3d-optim-ge}
\end{figure}

\tikzfig{figs/fidelity-length-ge}{%
Fidelity of first and last iteration of every pulse length.
The stable region above 
}{fig:fidelity-length-ge}{\textwidth}{15em}

Further analysis is done on pulses with lengths \SI{4.25}{\nano\second}, \SI{6.0}{\nano\second}, \SI{8.0}{\nano\second}, \SI{10.0}{\nano\second}, \SI{20.0}{\nano\second} and \SI{30.0}{\nano\second}.
The optimised pulse shapes \(\text{Re}(\Omega)\) and \(\text{Im}(\Omega)\) are plotted in \cref{fig:pulse_shape} together with the guess pulses. Pulses longer than \ns{20} require only fine adjustments to the Blackman guess pulse while shorter pulses have an imaginary part which is maximised for the whole duration of the pulse.

\begin{figure}[ht]
\centering
\foreach \n/\capn [count=\ni] in {{4,25}/{4.25},{6,0}/{6.0},{8,0}/{8.0},{10,0}/{10.0},{20,0}/{20.0},{30,0}/{30.0}}{
	\subcaptionbox{Pulse length \capn{} ns}{
		\centering
		\setlength\figureheight{9.75em}
		\setlength\figurewidth{0.45\textwidth}
		\input{figs/pulse_shape_\n_Real.tikz}
		\input{figs/pulse_shape_\n_Imag.tikz}
	}%
	\ifnum\ni=6%
	        %
	\else%
		\hfill
	\fi%
}
\caption{Optimised pulse shapes and guess pulses for pulse lengths 
\textbf{(a)} \SI{4.25}{\nano\second}, 
\textbf{(b)} \SI{6.0}{\nano\second}, 
\textbf{(c)} \SI{8.0}{\nano\second}, 
\textbf{(d)} \SI{10.0}{\nano\second}, 
\textbf{(e)} \SI{20.0}{\nano\second}, 
and \textbf{(f)} \SI{30.0}{\nano\second}.
Short pulses (\(<\)\ns{10}) change substantially from the starting Blackman shape while long pulses (\(>\)\ns{20}) only require fine adjustments.}
\label{fig:pulse_shape}
\end{figure}

The spectrum of \(\Omega(t)\) in the lab frame \(\Omega(t)\ex^{i\omega_{01} t}\) is shown in \cref{fig:pulse_spectrum}. For all pulse lengths there is a peak centered roughly at \(\omega_{01}\) and the width of the peak becomes narrower for longer pulses. For the pulse length \ns{30} there is almost no support at \(\omega_{02}\) and \(\omega_{12}\).

\begin{figure}[ht]
\centering
\foreach \n/\capn [count=\ni] in {{4,25}/{4.25},{6,0}/{6.0},{8,0}/{8.0},{10,0}/{10.0},{20,0}/{20.0},{30,0}/{30.0}}{
	\subcaptionbox{Pulse length \capn{} ns}{
		\centering
		\setlength\figureheight{10em}
		\setlength\figurewidth{0.47\textwidth}
		\input{figs/pulse_spectrum_\n.tikz}
	}%
	\ifnum\ni=6%
	        %
	\else%
		\hfill
	\fi%
}
\caption{Pulse spectrum of the complex pulses in \cref{fig:pulse_shape}. The vertical lines indicate (from left to right) \(\omega_{02}\), \(\omega_{12}\), \(\omega_{01}\).}
\label{fig:pulse_spectrum}
\end{figure}


\begin{figure}[ht]
\centering
\foreach \n/\capn [count=\ni] in {{4,25}/{4.25},{6,0}/{6.0},{8,0}/{8.0},{10,0}/{10.0},{20,0}/{20.0},{30,0}/{30.0}}{
	\subcaptionbox{Pulse length \capn{} ns}{
		\centering
		\setlength\figureheight{15em}
		\setlength\figurewidth{0.47\textwidth}
		\input{figs/qubit_occ_\n.tikz}
	}%
	\ifnum\ni=6%
	        %
	\else%
		\hfill
	\fi%
}
\caption{Energy level occupation over time for different lengths of optimised pulses.}
\label{fig:qubit_occupation}
\end{figure}

\begin{figure}[ht]
\centering
\foreach \n/\capn [count=\ni] in {{4,25}/{4.25},{6,0}/{6.0},{8,0}/{8.0},{10,0}/{10.0},{20,0}/{20.0},{30,0}/{30.0}}{
	\subcaptionbox{Pulse length \capn{} ns}{
		\centering
		\setlength\figureheight{0.30\textwidth}
		\setlength\figurewidth{0.30\textwidth}
		\input{figs/hinton_\n.tikz}
	}%
	\ifnum\ni=6%
	        %
	\else%
		\hfill
	\fi%
}
\caption{Hinton diagram of \(\op{\psi(T)}\)}
\label{fig:hinton}
\end{figure}


\begin{figure}[ht]
\centering
\foreach \n/\capn [count=\ni] in {{4,25}/{4.25},{6,0}/{6.0},{8,0}/{8.0},{10,0}/{10.0},{20,0}/{20.0},{30,0}/{30.0}}{
	\subcaptionbox{Pulse length \capn{} ns}{\includegraphics[width=0.3\linewidth]{figs/bloch_evolution_\n.png}}%
	\ifnum\ni=6%
	        %
	\else%
		\hfill
	\fi%
}
\caption{Time dynamics on the Bloch sphere for different lengths of optimised pulses.}
\label{fig:bloch_evolution}
\end{figure}


\section{\texorpdfstring{\boldmath\(\ket{0}\rightarrow\ket{2}\)}{0 -> 2} state transfer}

\begin{figure}
    \centering
    \includegraphics[width=0.7\linewidth]{figs/3d-optim-gf.png}
    \caption{Fidelity during optimisations for every pulse length (ns).}
    \label{fig:3d-optim-gf}
\end{figure}

\tikzfig{figs/fidelity-length-gf}{
Fidelity of first and last iteration of every pulse length.
}{fig:fidelity-length-gf}{0.8\textwidth}{15em}

\begin{figure}[ht]
\centering
\foreach \n/\capn [count=\ni] in {{22,0}/{22.0},{24,0}/{24.0},{26,0}/{26.0},{28,0}/{28.0},{29,0}/{29.0},{55,0}/{55.0}}{
	\subcaptionbox{Pulse length \capn{} ns}{
		\centering
		\setlength\figureheight{10em}
		\setlength\figurewidth{0.45\textwidth}
		\input{figs/pulse_shape_gf_\n_Real.tikz}
		\input{figs/pulse_shape_gf_\n_Imag.tikz}
	}%
	\ifnum\ni=6%
	        %
	\else%
		\hfill
	\fi%
}
\caption{Pulse shapes.}
\label{fig:pulse_shape_gf}
\end{figure}


\begin{figure}[ht]
\centering
\foreach \n/\capn [count=\ni] in {{22,0}/{22.0},{24,0}/{24.0},{26,0}/{26.0},{28,0}/{28.0},{29,0}/{29.0},{55,0}/{55.0}}{
	\subcaptionbox{Pulse length \capn{} ns}{
		\centering
		\setlength\figureheight{10em}
		\setlength\figurewidth{0.47\textwidth}
		\input{figs/pulse_spectrum_gf_\n.tikz}
	}%
	\ifnum\ni=6%
	        %
	\else%
		\hfill
	\fi%
}
\caption{Pulse spectrum}
\label{fig:pulse_spectrum_gf}
\end{figure}


\begin{figure}[ht]
\centering
\foreach \n/\capn [count=\ni] in {{22,0}/{22.0},{24,0}/{24.0},{26,0}/{26.0},{28,0}/{28.0},{29,0}/{29.0},{55,0}/{55.0}}{
	\subcaptionbox{Pulse length \capn{} ns}{
		\centering
		\setlength\figureheight{15em}
		\setlength\figurewidth{0.45\textwidth}
		\input{figs/qubit_occ_gf_\n.tikz}
	}%
	\ifnum\ni=6%
	        %
	\else%
		\hfill
	\fi%
}
\caption{Energy level occupation over time for different lengths of optimised pulses.}
\label{fig:qubit_occupation_gf}
\end{figure}

\begin{figure}[ht]
\centering
\foreach \n/\capn [count=\ni] in {{22,0}/{22.0},{24,0}/{24.0},{26,0}/{26.0},{28,0}/{28.0},{29,0}/{29.0},{55,0}/{55.0}}{
	\subcaptionbox{Pulse length \capn{} ns}{
		\centering
		\setlength\figureheight{0.30\textwidth}
		\setlength\figurewidth{0.30\textwidth}
		\input{figs/hinton_gf_\n.tikz}
	}%
	\ifnum\ni=6%
	        %
	\else%
		\hfill
	\fi%
}
\caption{Hinton diagram of \(\op{\psi(T)}\)}
\label{fig:hinton_gf}
\end{figure}


\end{document}
