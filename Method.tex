\documentclass[main.tex]{subfiles}
\begin{document}

\chapter{Method}
In this chapter Krotov's method will be briefly introduced along with an implementation in Python. Then the numerical experiments will be presented.



\section{Krotov's Method for quantum optimal control}
Krotov's method fundamentally relies on the variational principle to \minimize{} a functional \(J\qty[\qty{\ket{\phi^{(i)}_k(t)}},\qty{\epsilon_l^{(i)}(t)}]\) where the constraints are included as Lagrange multipliers \cite{goerz_krotov:_2019}. A detailed explanation of this functional when the method is applied to quantum systems can be found in \cite{reich_monotonically_2012}. 

\section{Krotov: the Python package}
A Python implementation of the Krotov package is available at \url{https://krotov.readthedocs.io/en/latest/}. It provides simple functions and objects to
Test

\section{Optimisation Experiments}
In this chapter the numerical optimisation experiments are presented and motivated.

Optimisation will realise state transfers

4 gigasamples/s

Amplitude constraint (with pi pulse calibration)

Guess pulses half amplitude (actually blackman pulses)

convergence criteria
fidelity \(F\)
change between iterations falls below a certain critera \(\Delta F\)

\subsection{\texorpdfstring{\boldmath\(\ket{0}\rightarrow\ket{1}\)}{0 -> 1} state transfer}
Pulse shapes were optimised with varying lengths from \SI{4.25}{\nano\second} to \SI{30}{\nano\second} with convergence criteria \(F>0.99999\) or \(\Delta F < \num{e-07}\). Step size \(\lambda = \frac{1}{\frac{1}{2}A_{m}}\)

\subsection{\texorpdfstring{\boldmath\(\ket{0}\rightarrow\ket{2}\)}{0 -> 2} state transfer}
Pulse shapes were optimised with varying lengths from \SI{22}{\nano\second} to \SI{30}{\nano\second} with convergence criteria \(F>0.99999\) or \(\Delta F < \num{e-09}\). Step size \(\lambda = \frac{1}{2A_{m}}\)

\end{document}
