\documentclass[main.tex]{subfiles}
\begin{document}

\chapter{Method}
In this chapter Krotov's method for quantum optimal control will be briefly introduced and an implementation as a Python package.
Then the numerical experiments will be explained.

\section{Krotov's Method for quantum optimal control}
Krotov's method fundamentally relies on the variational principle to minimize a functional \[ J\qty[\qty{\ket{\phi^{(i)}_k(t)}},\qty{\epsilon_l^{(i)}(t)}] \] where the constraints are included as Lagrange multipliers \cite{goerz_krotov:_2019}.
A detailed explanation of this functional when the method is applied to quantum systems can be found in \autocite{reich_monotonically_2012}. 

\section{Krotov: the Python package}
A Python implementation of the Krotov package is available at \url{https://krotov.readthedocs.io/en/latest/}. It provides simple functions and objects to


\section{Optimization Experiments}
In this chapter the numerical optimization experiments are presented and motivated.
\subsection{Hamiltonian}
To test the method, the anharmonic oscillator in~\cref{eq:resonator-hamiltonian} will be chosen as it is often used as the physical realisation of a qubit.
Throughout this thesis, such a system will be referred to as a qubit even though it has more than two energy levels.
Consequently, the resonance frequency of the qubit \( \omega_{01} \) refers to the transition between \( \ket{0} \) and \( \ket{1} \).
To induce transitions between these states, control terms are added to~\cref{eq:resonator-hamiltonian}
\begin{equation}
    \Ham = \omega_{01} \au\ad - \frac{\kappa}{2} \qty(\au\ad)^2 + \Omega(t)\ex^{i\omega_{01}t}\ad + \Omega^*(t)\ex^{-i\omega_{01}t}\au
    \label{eq:system-hamiltonian}
\end{equation}
where \( \Omega(t) \) is the complex amplitude of the driving pulse.
Looking at the Hamiltonian above it can be argued that it can be written in the form in \cref{eq:optimal-control-hamiltonian} with \( u_0(t) = \Omega(t)\ex^{i\omega_{01}t}, u_1(t) = \Omega^*(t)\ex^{-i\omega_{01}t} \).
However, there are two problems that need to be adressed.
Firstly, the oscillating factors will require an unnecessarily fine time discretization of the pulses.
Secondly, the Krotov package expects real-valued pulse amplitudes as inputs.
The first problem can be avoided by transforming the Hamiltonian into the interaction picture.
Choosing \( H_0 = \omega_{01} \au\ad \), \cref{eq:system-hamiltonian} transforms\footnote{Full derivation is shown in \cref{sec:appendix-interaction-picture}} into
\begin{equation}
    \Ham \rightarrow~ \Ham_R =  -\frac{\kappa}{2}\qty(\au\ad)^2 + \Omega(t)\ad + \Omega^*(t)\au.
    \label{eq:system-hamiltonian-rotating}
\end{equation}


\subsection{Optimization Setup}
Optimization will realise state transfers
\(\text{Re}(\Omega)\) and \(\text{Im}(\Omega)\) drive rotations around the x-axis and y-axis respectively (in the Bloch sphere). 
Rotating frame

4 gigasamples/s

Amplitude constraint (with pi pulse calibration)

Guess pulses half amplitude (actually blackman pulses)

convergence criteria
fidelity \(F\)
change between iterations falls below a certain critera \(\Delta F\)

\subsection{\texorpdfstring{\boldmath\(\ket{0}\rightarrow\ket{1}\)}{0 -> 1} state transfer}
Pulse shapes were optimized with varying lengths from \SI{4.25}{\nano\second} to \SI{30}{\nano\second} with convergence criteria \(F>0.99999\) or \(\Delta F < \num{e-07}\). Step size \(\lambda = \frac{1}{\frac{1}{2}A_{m}}\)

\subsection{\texorpdfstring{\boldmath\(\ket{0}\rightarrow\ket{2}\)}{0 -> 2} state transfer}
Pulse shapes were optimized with varying lengths from \SI{22}{\nano\second} to \SI{30}{\nano\second} with convergence criteria \(F>0.99999\) or \(\Delta F < \num{e-09}\). Step size \(\lambda = \frac{1}{2A_{m}}\)

\end{document}
