\documentclass[main.tex]{subfiles}
\begin{document}

\chapter{Introduction}
Quantum computing is starting to appear in commercial products~\cite{santos_ibm_2016}, however, despite a lot of progress~\cite{preskill_quantum_2018}, there are still a lot of challenges to be solved before large scale quantum computers become commonplace.
Unlike classical computers, where the transistors' on and off state is dictated by the stream of many electrons, quantum computers rely on single or very few number of particles which make the quantum states very delicate~\cite{gottesman_introduction_2009}.

One of the key technologies to keep these states in quantum computing is quantum error correction (QEC)~\cite{gottesman_introduction_2009}, which is a way to retain the quantum information in a qubit by introducing redundancy into the physical system.
\citeauthor{leghtas_hardware-efficient_2013}~\cite{leghtas_hardware-efficient_2013},~\citeauthor{mirrahimi_dynamically_2014}~\cite{mirrahimi_dynamically_2014} propose and~\citeauthor{ofek_extending_2016}~\cite{ofek_extending_2016} demonstrate a method to encode the quantum information in a clever basis in quantum harmonic resonators.

To further clarify, in this scheme the quantum information of a qubit is carefully encoded and decoded into a resonator by applying optimized microwave pulses to the qubit and resonator system which will realise state transfers in both these systems. This thesis will focus on how to numerically optimize encoding and decoding pulses using Krotov's method~\cite{reich_monotonically_2012}, a gradient ascent based optimization algorithm available publicly as a ready-to-use Python package~\cite{goerz_krotov:_2019}.

\section{Purpose}
The purpose of this thesis is to numerically optimize microwave pulses to transfer the quantum information in a qubit to a resonator. This will be done using the Krotov Python package~\cite{goerz_krotov:_2019}.
% TODO add subgoals

\section{Limitations of the thesis}
In this thesis, the limitations will be the following:
\begin{itemize}
    \item The quantum systems are assumed to be closed systems without any interaction with the environment.
    \item The step size in Krotov's method will be constant throughout the optimizations.
    \item Single optimization goal (although multiple simultaneous are possible).
\end{itemize}


\end{document}
