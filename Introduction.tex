\documentclass[main.tex]{subfiles}
\begin{document}

\chapter{Introduction}
Quantum computing is starting to appear in commercial products~\cite{santos_ibm_2016}, however, despite progress~\cite{preskill_quantum_2018}, there are still a lot of challenges to be solved before large scale quantum computers become commonplace.
Unlike classical computers, where the transistors' on and off state is dictated by the stream of many electrons, quantum computers rely on single or very few number of particles which make the quantum states very delicate~\cite{gottesman_introduction_2009}.
One of the key technologies to keep these states stable is quantum error correction (QEC)~\cite{gottesman_introduction_2009}, which is a way to protect quantum information from noise. One way to do this is to introduce redundancy into the physical system and spread the information to highly entangled states using \emph{quantum error correcting codes}.

\citeauthor{leghtas_hardware-efficient_2013}~\cite{leghtas_hardware-efficient_2013},~\citeauthor{mirrahimi_dynamically_2014}~\cite{mirrahimi_dynamically_2014} propose and~\citeauthor{ofek_extending_2016}~\cite{ofek_extending_2016} demonstrate a method to encode the quantum information in a ``cat code'' in quantum harmonic resonators.
The theory of the cat code presented in \cref{sec:cat-code}.
To further clarify the encoding scheme, the quantum information of a superconducting qubit is carefully encoded into a cavity resonator by simultaneously driving the qubit and resonator system with microwave pulses.
This realises the desired state transfers in both the qubit and the resonator.
~\citeauthor{ofek_extending_2016} numerically optimize the microwave pulse shapes using the GRAPE algorithm~\cite{khaneja_optimal_2005} which uses gradient ascent while simulating the system to update the pulse shapes until the state transfer is realised.

There is no source code available for this method which provides an opportunity to reproduce this result with an easy-to-use open source software package and hopefully provide reference for future implementations.

\section{Purpose of the Thesis}
The purpose of this thesis is to, in simulation, numerically optimize microwave pulses to encode the quantum information of a qubit to the cat code as a proof of concept.
The pulses will be optimized using Krotov's method~\cite{reich_monotonically_2012}, a gradient-based optimization algorithm available publicly as a ready-to-use Python package~\cite{goerz_krotov:_2019}.

The scope of the thesis is limited to
\begin{itemize}
    \item closed quantum systems without any interaction with the environment,
    \item assessing the performance of the method without fully understanding the underlying explanations for the results and
    \item only finding \emph{encoding}-pulses (decoding-pulses introduce additional complexity).
\end{itemize}

\end{document}
